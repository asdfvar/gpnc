\documentclass{article}
\usepackage[in]{fullpage}
\usepackage{amsmath}
\usepackage{amsfonts}
%
\begin{document}
%
% paper title
\title{Discrete Fourier Transform}
%\author{Craig~Euler}
%
\maketitle
%
\section {Fast Fourier Transform}
%
The Fast Fourier Transform (FFT) is a fast implementation of the Discrete Fourier Transform (DFT) where the DFT is defined as:
%
\begin {equation} \label {eq:dft}
X_k = \sum_{p = 0}^{N-1} (x_p \omega^{k p})
\end {equation}
%
where
%
\begin {equation} \label {eq:twiddle}
\omega_N = e^{-2 \pi \mathrm {i} / N}
\end {equation}
%
$\forall x_p \in \mathbb {C}$. Having $N$ multiplies and $N-1$ adds over $N$ points, the DFT has a total number of $2 N^2 - N$ add/multiplies or an order of complexity $O(N^2)$. The FFT exploits two facts about the twiddle factors:
%
\begin {equation} \label {half}
\begin {aligned}
\omega_N^{2m} &= e^{-2 \pi \mathrm {i} (2m) / N}    \\
              &= e^{-2 \pi \mathrm {i} m / (N / 2)} \\
\omega_N^{2m} &= \omega_{N/2}^m
\end {aligned}
\end {equation}
%
and symmetry (which is apparent on the unit circle)
%
\begin {equation} \label {symmetry}
\begin {aligned}
\omega_N^{k + N/2} &= e^{-2 \pi \mathrm {i} (k + N/2) / N} \\
             &= \left (e^{-2 \pi \mathrm {i} (k) / N} \right) \left (e^{-2 \pi \mathrm {i} (N/2) / N} \right) \\
             &= \left (e^{-2 \pi \mathrm {i} (k) / N} \right) \left (-1 \right) \\
\omega_N^{k + N/2} &= -\omega_{N}^{k}
\end {aligned}
\end {equation}
%
which means, from setting $p \rightarrow 2m$, splitting the sum over the even and odd terms, and using (\ref{half}),
%
\begin {equation} \label {eq:xk}
\begin {aligned}
X_k &= \sum_{p = 0}^{N-1} (x_p \omega_N^{k p}) \\
    &= \sum_{m = 0}^{N/2-1} (x_{2m} \omega_N^{k 2 m} + x_{2m+1} \omega_N^{k (2 m + 1)}) \\
    &= \sum_{m = 0}^{N/2-1} (x_{2m} \omega_{N/2}^{k m}) + \omega_N^{k} \sum_{m = 0}^{N/2-1} (x_{2m+1} \omega_{N/2}^{km}) \\
X_k &= DFT_k(x_{even}; N/2) + \omega_N^{k} DFT_k(x_{odd}; N/2)
\end {aligned}
\end {equation}
%
is now broken up into two smaller DFT's.
Another observation, from using (\ref{symmetry}), symmetry is observed:
%
\begin {equation} \label {eq:xksym}
\begin {aligned}
X_{k + N/2} &= \sum_{p = 0}^{N-1} (x_p \omega_N^{(k + N/2) p}) \\
            &= \sum_{p = 0}^{N-1} (x_p \omega_N^{k p}) (\omega_N^{p N/2}) \\
            &= \sum_{p = 0}^{N-1} (x_p \omega_N^{k p}) (-1)^p \\
            &= \sum_{m = 0}^{N/2-1} (x_{2m} \omega_{N/2}^{k m} (-1)^{2m}) + \omega_N^{k} \sum_{m = 0}^{N/2-1} (x_{2m+1} \omega_{N/2}^{km} (-1)^{2m+1}) \\
            &= \sum_{m = 0}^{N/2-1} (x_{2m} \omega_{N/2}^{k m}) - \omega_N^{k} \sum_{m = 0}^{N/2-1} (x_{2m+1} \omega_{N/2}^{km}) \\
X_{k + N/2} &= DFT_k(x_{even}; N/2) - \omega_N^{k} DFT_k(x_{odd}; N/2).
\end {aligned}
\end {equation}
%
Re-writing (\ref{eq:xk}) and (\ref{eq:xksym}) together for convenience:
%
\begin {equation}
\begin {aligned}
&X_k         &= DFT_k(x_{even}; N/2) + \omega_N^{k} DFT_k(x_{odd}; N/2) \\
&X_{k + N/2} &= DFT_k(x_{even}; N/2) - \omega_N^{k} DFT_k(x_{odd}; N/2).
\end {aligned}
\end {equation}
%

With $N/2$ multiplies and $N/2 - 1$ adds from each of the smaller DFT's over $N/2$ points applied twice followed by an additional $N/2$ multiplies from the twiddle factors (on the outside of the sum terms) and $N$ adds between the two DFT's across all $k$ results in a total of $2 * N/2 * (N/2 + N/2 - 1) + N/2 + N$ or $N^2 + N/2$ add/multiplies roughly halving the number of operations ($2 N^2 - N \rightarrow N^2 + N/2 = (2 N^2 - N) / 2 + N$).

Defining $F_0 (N)$ to be the number of adds/multiplies (FLOPs) of the DFT of $N$ points:
%
\begin {equation} \label {eq:F0}
F_0 (N) := 2 N^2 - N
\end {equation}
%
and if $F_n(N)$ is the $n^{th}$ ``decimation'' of the DFT, then, at the first decimation, there will be $2 F_0(N/2)$ smaller DFT's followed by $N/2$ adds and $N$ multiplies ($3/2 N$ add/multiplies), or
%
\begin {equation} \label {eq:F1}
F_1 (N) := 2 F_0 (N/2) + 3/2 N
\end {equation}
%
Interestingly, (\ref{eq:F1}) can be expressed in terms of $F_0(N)$ as
%
\begin {equation}
\begin {aligned}
F_1 (N) &= 2 F_0 (N/2) + 3/2 N \\
        &= N^2 + N/2 \\
        &= (2 N^2 - N)/2 + N \\
F_1 (N) &= F_0(N)/2 + N
\end {aligned}
\end {equation}
%
and whether this generalizes or not is for another time.
It does, however, follow that (\ref{eq:F1}) generalizes for all decimations of $n$:
%
\begin {equation} \label {eq:Fn}
\begin {aligned}
F_n (N) &= 2 F_{n - 1} (N/2) + 3/2 N
\end {aligned}
\end {equation}
%
Then, inductively repeating the number of ``decimations'', $F(N)$ reduces down to
%
\begin {equation} \label {eq:Fb}
\begin {aligned}
F_\beta (N) &= 2^\beta F_{\alpha - \beta} (N/2^\beta) + 3/2 \beta N
\end {aligned}
\end {equation}
%
hence
%
\begin {equation} \label {eq:Fa}
F_\alpha (N) = 2^\alpha F_0 (N/2^\alpha) + 3/2 \alpha N.
\end {equation}
%
Expanding on (\ref{eq:Fa}),
%
\begin {equation}
\begin {aligned}
F_\alpha (N) &= 2^\alpha F_0 (N/2^\alpha) + 3/2 \alpha N \\
F_\alpha (N) &= N^2/2^{\alpha - 1} + 3/2 \alpha N - N \\
F_\alpha (N) &= N (N/2^{\alpha - 1} + 3/2 \alpha - 1)
\end {aligned}
\end {equation}
%
and if $N = 2^\alpha$, then
%
\begin {equation} \label {eq:Fradix2}
\begin {aligned}
F_\alpha (2^\alpha) &= 2^{\alpha} (1 + 3/2 \alpha).
\end {aligned}
\end {equation}
%
Notice that $log_2 (N = 2^\alpha) = \alpha$ showing that (\ref{eq:Fradix2}) is proportional to $N log_2 (N)$ for radix $2$ values of $N$.
%
\section {Conjugate Symmetry}
%
Set $p \rightarrow m + N/2 - 1$, then (\ref{eq:dft}) becomes
%
\begin {equation} \label {eq:redef}
\begin {aligned}
X_k &= \sum_{p = 0}^{N-1} (x_p \omega^{k p}) \\
    &= \sum_{m = -N/2 + 1}^{N/2} (x_{m + N/2 - 1} \omega^{k (m + N/2 - 1)}) \\
X_k &= \omega^{k (N/2 - 1)} \sum_{m = -N/2 + 1}^{N/2} (x_{m + N/2 - 1} \omega^{k m}).
\end {aligned}
\end {equation}
%
Then, define $z_m := x_{m + N/2 - 1}$ so that (\ref{eq:redef}) becomes
%
\begin {equation}
\begin {aligned}
X_k &= \omega^{k (N/2 - 1)} \sum_{m = -N/2 + 1}^{N/2} (x_{m + N/2 - 1} \omega^{k m}) \\
X_k &= \omega^{k (N/2 - 1)} \sum_{m = -N/2 + 1}^{N/2} (z_m \omega^{k m})
\end {aligned}
\end {equation}
%
effectively changing the indices to $(-N/2, N/2]$.
Then, without loss of generality on the index of $x$, the DFT definition can be tweaked a bit to range from $(-N/2, N/2]$.
If $x_p \in \mathbb {R}$ $\forall p$ and given
%
\begin {equation}
\left (\omega_N^{k} \right)^*
 = \left (e^{-2 \pi \mathrm {i} / N (k)} \right)^* 
 = e^{-2 \pi \mathrm {i} / N (-k)} 
 = \omega_{N}^{-k}
\end {equation}
%
then
%
\begin {equation}
\begin {aligned}
X_k^* = \left( \omega^{k (N/2 - 1)} \sum_{p = -N/2+1}^{N/2} (x_p \omega_N^{k p}) \right)^*
      = \omega^{(-k) (N/2 - 1)} \sum_{p = -N/2+1}^{N/2} (x_p \omega_N^{(-k) p})
      = X_{-k}
\end {aligned}
\end {equation}
%
showing conjugate symmetry for $\mathbb{R} \rightarrow \mathbb{C}$ transforms.

\section {$\mathbb{R} \rightarrow \mathbb{C}$ Transforms}
%
As a result of conjugate symmetry, $\mathbb{R} \rightarrow \mathbb{C}$ transforms only need to account for roughly half of the data.

One approach is to explicitly develop an FFT implementation that is aware of $\mathbb{R} \rightarrow \mathbb{C}$ transforms and only compute half the result effectively saving space and processing.
However, there is an interesting trick.
Starting from the definition of the DFT (\ref{eq:dft}) and multiplexing two real, independent data arrays $\{x\},\{y\} \in \mathbb{R}^n$ into $\{z\} \in \mathbb{C}^n$ such that $z_i = x_i + \mathrm {i} y_i$ $\forall x_i,y_i \in X,Y$:
%
\begin {equation}
\begin {aligned}
Z_k &= \sum_{p = 0}^{N-1} (z_p \omega^{k p}) = \sum_{p = 0}^{N-1} (x_p \omega^{k p}) + \mathrm{i} \sum_{p = 0}^{N-1} (y_p \omega^{k p}) \\
Z_k &= X_k + \mathrm{i} Y_k \\
\end {aligned}
\end {equation}
%
and so, by exploiting conjugate symmetry,
%
\begin {equation}
\begin {aligned}
\frac{1}{2} (Z_k + Z_{-k}^*) &= \frac{1}{2} (X_k + \mathrm{i} Y_k + (X_{-k} + \mathrm{i} Y_{-k})^*) = \frac{1}{2} (X_k + \mathrm{i} Y_k + X_k - \mathrm{i} Y_k) \\
\frac{1}{2} (Z_k + Z_{-k}^*) &= X_k
\end {aligned}
\end {equation}
%
and
%
\begin {equation}
\begin {aligned}
\frac{1}{2} (-\mathrm{i} Z_k + \mathrm{i} Z_{-k}^*) &= \frac{1}{2} (-\mathrm{i} X_k + Y_k + \mathrm{i} (X_{-k} + Y_{-k})^*) = \frac{1}{2} (-\mathrm{i} X_k + Y_k + \mathrm{i} X_k + Y_k) \\
\frac{1}{2} (-\mathrm{i} Z_k + \mathrm{i} Z_{-k}^*) &= Y_k
\end {aligned}
\end {equation}
%
\end{document}
