\documentclass{article}
\usepackage[in]{fullpage}
\usepackage{amsmath}
\usepackage{amsfonts}
%
\begin{document}
%
% paper title
\title{Fast Fourier Transform}
%\author{Craig~Euler}
%
\maketitle
%
The Fast Fourier Transform (FFT) is a fast implementation of the Discrete Fourier Transform (DFT) where the DFT is defined as:
%
\begin {equation} \label {eq:dft}
X_k = \sum_{p = 0}^{N-1} (x_p \omega^{k p})
\end {equation}
%
where
%
\begin {equation} \label {eq:twiddle}
\omega_N = e^{-2 \pi \mathrm {i} / N}
\end {equation}
%
$\forall x_p \in \mathbb {C}$. Having $N$ multiplies and $N-1$ adds over $N$ points, the DFT has a total number of $2 N^2 - N$ add/multiplies or an order of complexity $O(N^2)$. The FFT exploits two facts about the twiddle factors:
%
\begin {equation} \label {half}
\begin {aligned}
\omega_N^{2m} &= e^{-2 \pi \mathrm {i} (2m) / N}    \\
              &= e^{-2 \pi \mathrm {i} m / (N / 2)} \\
\omega_N^{2m} &= \omega_{N/2}^m
\end {aligned}
\end {equation}
%
and symmetry
%
\begin {equation} \label {symmetry}
\begin {aligned}
\omega_N^{k + N/2} &= e^{-2 \pi \mathrm {i} (k + N/2) / N} \\
             &= \left (e^{-2 \pi \mathrm {i} (k) / N} \right) \left (e^{-2 \pi \mathrm {i} (N/2) / N} \right) \\
             &= \left (e^{-2 \pi \mathrm {i} (k) / N} \right) \left (-1 \right) \\
\omega_N^{k + N/2} &= -\omega_{N}^{k}
\end {aligned}
\end {equation}
%
which means, from setting $p \rightarrow 2m$, splitting the sum over the even and odd terms, and using (\ref{half}),
%
\begin {equation} \label {eq:xk}
\begin {aligned}
X_k &= \sum_{p = 0}^{N-1} (x_p \omega_N^{k p}) \\
    &= \sum_{m = 0}^{N/2-1} (x_{2m} \omega_N^{k 2 m} + x_{2m+1} \omega_N^{k (2 m + 1)}) \\
    &= \sum_{m = 0}^{N/2-1} (x_{2m} \omega_{N/2}^{k m}) + \omega_N^{k} \sum_{m = 0}^{N/2-1} (x_{2m+1} \omega_{N/2}^{km}) \\
X_k &= DFT(x_{even}; N/2) + \omega_N^{k} DFT(x_{odd}; N/2)
\end {aligned}
\end {equation}
%
is now broken up into two smaller DFT's.
Another observation, from using (\ref{symmetry}), symmetry is observed:
%
\begin {equation} \label {eq:xksym}
\begin {aligned}
X_{k + N/2} &= \sum_{p = 0}^{N-1} (x_p \omega_N^{(k + N/2) p}) \\
            &= \sum_{p = 0}^{N-1} (x_p \omega_N^{k p}) (\omega_N^{p N/2}) \\
            &= \sum_{p = 0}^{N-1} (x_p \omega_N^{k p}) (-1)^p \\
            &= \sum_{m = 0}^{N/2-1} (x_{2m} \omega_{N/2}^{k m} (-1)^{2m}) + \omega_N^{k} \sum_{m = 0}^{N/2-1} (x_{2m+1} \omega_{N/2}^{km} (-1)^{2m+1}) \\
            &= \sum_{m = 0}^{N/2-1} (x_{2m} \omega_{N/2}^{k m}) - \omega_N^{k} \sum_{m = 0}^{N/2-1} (x_{2m+1} \omega_{N/2}^{km}) \\
X_{k + N/2} &= DFT(x_{even}; N/2) - \omega_N^{k} DFT(x_{odd}; N/2)
%
\end {aligned}
\end {equation}
%
Where the difference between (\ref{eq:xk}) and (\ref{eq:xksym}) is a "-" between the two components.
With $N/2$ multiplies and $N/2 - 1$ adds from each of the smaller DFT's over $N/2$ points applied twice followed by an additional $N/2$ multiplies from the twiddle factors (on the outside of the sum terms) and $N$ adds between the two DFT's across all $k$ results in a total of $2 * N/2 * (N/2 + N/2 - 1) + N/2 + N$ or $N^2 + N/2$ add/multiplies roughly halving the number of operations ($2 N^2 - N \rightarrow N^2 + N/2 = (2 N^2 - N) / 2 + N$).
If the number of points is radix 2, then the process can be done $log_2 (N)$ times.
%
\section {Conjugate Symmetry}
%
Set $p \rightarrow m + N/2 - 1$, then (\ref{eq:dft}) becomes
%
\begin {equation}
\begin {aligned}
X_k &= \sum_{p = 0}^{N-1} (x_p \omega^{k p}) \\
    &= \sum_{m = -N/2 + 1}^{N/2} (x_{m + N/2 - 1} \omega^{k (m + N/2 - 1)}) \\
X_k &= \omega^{k N/2 - k} \sum_{m = -N/2 + 1}^{N/2} (x_{m + N/2 - 1} \omega^{k m})
\end {aligned}
\end {equation}
%
effectively changing the indices.
Then, without loss of generality on the index of $x$, the DFT definition can be tweaked a bit to range from $(-N/2, N/2]$.
If $x_p \in \mathbb {R}$ $\forall p$ and given
%
\begin {equation}
\begin {aligned}
\left (\omega_N^{k} \right)^* &= \left (e^{-2 \pi \mathrm {i} / N (k)} \right)^* \\
                              &= e^{-2 \pi \mathrm {i} / N (-k)} \\
\left (\omega_N^{k} \right)^* &= \omega_{N}^{-k}
\end {aligned}
\end {equation}
%
then
%
\begin {equation}
\begin {aligned}
X_k^* = \sum_{p = -N/2+1}^{N/2} (x_p \omega_N^{k p})^*
      = \sum_{p = -N/2+1}^{N/2} (x_p \omega_N^{-k p})
      = X_{-k}
\end {aligned}
\end {equation}
%
showing conjugate symmetry for $\mathbb{R} \rightarrow \mathbb{C}$ transforms.
%
\end{document}
