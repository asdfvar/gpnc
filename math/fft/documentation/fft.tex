\documentclass{article}
\usepackage[in]{fullpage}
\usepackage{amsmath}
\usepackage{amsfonts}
%
\begin{document}
%
% paper title
\title{Fast Fourier Transform}
%\author{Craig~Euler}
%
\maketitle
%
The Fast Fourier Transform (FFT) is a fast implementation of the Discrete Fourier Transform (DFT) where the DFT is defined as:
%
\begin {equation} \label {eq:dft}
X_k = \sum_{p = 0}^{N-1} (x_p \omega^{k p})
\end {equation}
%
where
%
\begin {equation} \label {eq:twiddle}
\omega_N = e^{-2 \pi \mathrm {i} / N}
\end {equation}
%
$\forall x_p \in \mathbb {C}$. The FFT exploits two facts about the twiddle factors:
%
\begin {equation}
\begin {aligned}
\omega_N^{2m} &= e^{-2 \pi \mathrm {i} (2m) / N}    \\
              &= e^{-2 \pi \mathrm {i} m / (N / 2)} \\
              &= \omega_{N/2}^m
\end {aligned}
\end {equation}
%
and symmetry
%
\begin {equation}
\begin {aligned}
\omega_N^{k + N/2} &= e^{-2 \pi \mathrm {i} (k + N/2) / N} \\
             &= \left (e^{-2 \pi \mathrm {i} (k) / N} \right) \left (e^{-2 \pi \mathrm {i} (N/2) / N} \right) \\
             &= \left (e^{-2 \pi \mathrm {i} (k) / N} \right) \left (-1 \right) \\
             &= -\omega_{N}^{k}
\end {aligned}
\end {equation}
%
which means
%
\begin {equation} \label {eq:xk}
\begin {aligned}
X_k &= \sum_{p = 0}^{N-1} (x_p \omega_N^{k p}) \\
    &= \sum_{m = 0}^{N/2-1} (x_{2m} \omega_N^{k 2 m} + x_{2m+1} \omega_N^{k (2 m + 1)}) \\
    &= \sum_{m = 0}^{N/2-1} (x_{2m} \omega_{N/2}^{k m}) + \omega_N^{k} \sum_{m = 0}^{N/2-1} (x_{2m+1} \omega_{N/2}^{km})
\end {aligned}
\end {equation}
%
for $2m = p$.
%
\begin {equation} \label {eq:xksym}
\begin {aligned}
X_{k + N/2} &= \sum_{p = 0}^{N-1} (x_p \omega_N^{(k + N/2) p}) \\
            &= \sum_{p = 0}^{N-1} (x_p \omega_N^{k p}) (\omega_N^{p N/2}) \\
            &= \sum_{p = 0}^{N-1} (x_p \omega_N^{k p}) (-1)^p \\
            &= \sum_{m = 0}^{N/2-1} (x_{2m} \omega_{N/2}^{k m} (-1)^{2m}) + \omega_N^{k} \sum_{m = 0}^{N/2-1} (x_{2m+1} \omega_{N/2}^{km} (-1)^{2m+1}) \\
            &= \sum_{m = 0}^{N/2-1} (x_{2m} \omega_{N/2}^{k m}) - \omega_N^{k} \sum_{m = 0}^{N/2-1} (x_{2m+1} \omega_{N/2}^{km})
%
\end {aligned}
\end {equation}
%
Where the difference between (\ref{eq:xk}) and (\ref{eq:xksym}) is a "-" between the two components.
%
\vfill
%
For $\mathbb {R} \rightarrow \mathbb {C}$ transforms, Given
%
\begin {equation}
\begin {aligned}
\left (\omega_N^{k} \right)^* &= \left (e^{-2 \pi \mathrm {i} (k) / N} \right)^* \\
                              &= e^{-2 \pi \mathrm {i} (-k) / N} \\
                              &= \omega_{N}^{-k}
\end {aligned}
\end {equation}
%
if $x_p \in \mathbb {R}$, then
\begin {equation}
\begin {aligned}
X_k^* = \sum_{p = -N/2+1}^{N/2} (x_p \omega_N^{k p})^*
      = \sum_{p = -N/2+1}^{N/2} (x_p \omega_N^{-k p})
      = X_{-k}
\end {aligned}
\end {equation}
%
\end{document}
