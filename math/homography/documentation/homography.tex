\documentclass{article}
\usepackage[in]{fullpage}
\usepackage{amsmath}
\usepackage{amsfonts}
%
\begin{document}
%
% paper title
\title {Homography}
%\author{Craig~Euler}
%
\maketitle
%
\section {The Homography}
%
Words words words then
%
\begin {equation} \label {eq:homography}
H = d R + t n^t
\end {equation}
%
\section {Decomposition of the Homography}
%
The derivation of this decomposition is based on the paper ``Motion and Structure from Motion in a piecewise planar environment'' by Olivier Faugeras and Francis Lustman.

Start by computing the components of the SVD of \eqref {eq:homography} to obtain $H = U \Lambda V^t$. It follows that $\Lambda$ is also a homography with components:
%
\begin {equation} \label {eq:homography_p}
\Lambda = d' R' + t' n'
\end {equation}
%
where
%
\begin {equation}
\begin {aligned}
R &= s U R' V^t \\
t &= U t' \\
\hat {n} &= V \hat {n}' \\
d &= s d' \\
s &= det (U) det (V)
\end {aligned}
\end {equation}
%
Using the canonical basis $(e_1, e_2, e_3)$, $\hat {n}' = \hat{n}'_1 e_1 + \hat{n}'_2 e_2 + \hat{n}'_3 e_3$, \eqref {eq:homography_p} is written into its components:
%
\begin {equation} \label {eq:homography_components}
\lambda_i e_i = d' R' e_i + t' \hat{n}'_i
\end {equation}
%
where $\lambda_i$ is a component of the diagonal matrix $\Lambda$.
Multiply equation $i$ of \eqref {eq:homography_components} by $\hat{n}'_j$ and equation $j$ by $\hat{n}'_i$ and subtract to eliminate $t$ and obtain:
%
\begin {equation} \label {eq:eliminate_t}
d' R' (\hat{n}'_j e_i - \hat{n}'_i e_j) = \lambda_i \hat{n}'_j e_i - \lambda_j \hat{n}'_i e_j.
\end {equation}
%
Take the $L_2$ norm of \eqref {eq:eliminate_t}:
%
\begin {equation}
||d' R' (\hat{n}'_j e_i - \hat{n}'_i e_j)||^2 = ||\lambda_i \hat{n}'_j e_i - \lambda_j \hat{n}'_i e_j||^2
\end {equation}
%
Since $R'$ is a rotation and hence preserves the vector norm, it can be eliminated
%
\begin {equation}
\begin {aligned}
||d' (\hat{n}'_j e_i - \hat{n}'_i e_j)||^2 &= ||\lambda_i \hat{n}'_j e_i - \lambda_j \hat{n}'_i e_j||^2 \\
d'^2 (\hat{n}'^2_j + \hat{n}'^2_i) &= \lambda_i^2 \hat{n}'^2_j + \lambda^2_j \hat{n}'^2_i
\end {aligned}
\end {equation}
%
Re-ordering the terms produces the system of equations:
%
\begin {equation}
\label {eq:system_of_equations}
(d'^2 - \lambda_j^2) \hat{n}'^2_i + (d'^2 - \lambda_i^2) \hat{n}'^2_j = 0
\end {equation}
%
which means that the determinant must be zero:
%
\begin {equation}
(d'^2 - \lambda_1^2) (d'^2 - \lambda_2^2) (d'^2 - \lambda_3^2) = 0.
\end {equation}
%
Because $\lambda_1 > \lambda_2 > \lambda_3$ and \eqref {eq:system_of_equations}, it follows that $d' = \pm \lambda_2$.
%
Using the fact that $n'$ has a unit norm and \eqref {eq:system_of_equations}, it follows that
%
\begin {equation}
\label {eq:normal}
\left \{
\begin {aligned}
\hat{n}'_1 &= \epsilon_1 \sqrt {\frac {\lambda_1^2 - \lambda_2^2}{\lambda_1^2 - \lambda_3^2}} \\
\hat{n}'_2 &= 0 \\
\hat{n}'_3 &= \epsilon_3 \sqrt {\frac {\lambda_2^2 - \lambda_3^2}{\lambda_1^2 - \lambda_3^2}}
\end {aligned}
\right.
\end {equation}
%
where $\epsilon_1, \epsilon_3 = \pm 1$.

\noindent
\textbf {Case $d' > 0$:}

\noindent
because $\hat{n}'_2 = 0$ and $d' = \lambda_2$, if follows from from \eqref {eq:homography_components} that $R' e_2 = e_2$ and so $R'$ is a rotation matrix about $e_2$:
%
\begin {equation}
R' =
\begin {pmatrix}
cos \theta & 0 & -sin \theta \\
0          & 1 &    0        \\
sin \theta & 0 &  cos \theta
\end {pmatrix}
\end {equation}
%
Using \eqref {eq:eliminate_t} and \eqref {eq:normal},
%
\begin {equation}
\begin {aligned}
sin \theta &= (\lambda_1 - \lambda_3) \frac {\hat{n}'_1 \hat{n}'_3} {\lambda_2} \\
cos \theta &= \frac {\lambda_1 \hat{n}'^2_3 + \lambda_3 \hat{n}'^2_1} {\lambda_2}
\end {aligned}
\end {equation}
%
substituting the appropriate values into \eqref {eq:homography_components} yeilds
%
\begin {equation}
t' = (\lambda_1 - \lambda_3)
\begin {pmatrix}
\hat{n}'_1 \\
0 \\
-\hat{n}'_3 \\
\end {pmatrix}
\end {equation}

\noindent
\textbf {case $d' < 0$:}

\noindent
Then $R' e_2 = -e_2$ and so $R'$ can be written as
%
\begin {equation}
R' =
\begin {pmatrix}
cos \phi &  0 &  sin \phi \\
0        & -1 &  0        \\
sin \phi &  0 & -cos \phi
\end {pmatrix}
\end {equation}
%
with
%
\begin {equation}
\begin {aligned}
sin \phi &= \frac {\lambda_1 + \lambda_3} {\lambda_2} \hat{n}'_1 \hat{n}'_3 \\
cos \phi &= \frac {\lambda_3 \hat{n}'^2_1 - \lambda_1 \hat{n}'^2_3} {\lambda_2}
\end {aligned}
\end {equation}
%
And, similar to the process before,
%
\begin {equation}
t' = (\lambda_1 + \lambda_3)
\begin {pmatrix}
\hat{n}'_1 \\
0 \\
\hat{n}'_3
\end {pmatrix}
\end {equation}
%
\end {document}
